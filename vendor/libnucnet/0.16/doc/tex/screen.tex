%///////////////////////////////////////////////////////////////////////////////
% <file type="public">
%
% <license>
%   See the README_module.xml file for this module for copyright and license
%   information.
% </license>
%   <description>
%     <abstract>
%       This is the Webnucleo Report for screening and reverse rate
%       correction factors for libnucnet.
%     </abstract>
%     <keywords>
%       libnucnet Module, webnucleo report, screen, reverse, ratio
%     </keywords>
%   </description>
%
%   <authors>
%     <current>
%       <author userid="mbradle" start_date="2008/01/28" />
%     </current>
%     <previous>
%     </previous>
%   </authors>
%
%   <compatibility>
%     TeX (Web2C 7.4.5) 3.14159 kpathsea version 3.4.5
%   </compatibility>
%
% </file>
%///////////////////////////////////////////////////////////////////////////////
%
% This is a sample LaTeX input file.  (Version of 9 April 1986)
%
% A '%' character causes TeX to ignore all remaining text on the line,
% and is used for comments like this one.

\documentclass{article}    % Specifies the document style.

\usepackage[dvips]{graphicx}
\usepackage{hyperref}

\def\apj{Astrophys. J}

                           % The preamble begins here.
\title{Webnucleo Technical Report: Screening and Reverse Rate Correction
Factors in libnucnet}  % Declares the document's title.

\author{Bradley S. Meyer}
%\date{December 12, 1984}   % Deleting this command produces today's date.

\begin{document}           % End of preamble and beginning of text.

\maketitle                 % Produces the title.


This technical report describes how to provide routines to include
screening and reverse ratio correction factors for rate calculations
in libnucnet.

\section{Screening}

The dense electron gas present in plasmas can enhance the rate for a
thermonuclear reaction because the negative charge of the electrons screens
the positive charges of the interacting nuclei.  This makes the penetration
of the Coulomb barrier between the two positively charged reactants easier
and thus increases their interaction rate.
Expressions for screening exist in the literature and have been widely
employed in nucleosynthesis calculations (e.g., \cite{1982ApJ...258..696W}).
The question addressed here is how to implement screening in libnucnet.

One way to do this is to compute the forward and reverse nuclear reaction
rates for a zone using the libnucnet API routine
Libnucnet\_\_Zone\_\_computeRates().
Once the rates are computed, one can
then iterate over the reactions and apply the screening function to each
reaction.  Once the screening factor is computed, it is then applied to
the rates for the reaction by using the API routine
Libnucnet\_\_Zone\_\_replaceRatesForReaction().

As of version 0.2 of libnucnet, there is a more efficient way to include
screening in libnucnet calculations.  The user may supply a routine,
called a Libnucnet\_\_Net\_\_screening\_function in libnucnet.  Libnucnet
will use this routine to compute screening.

To supply a screening function, the user first writes a routine with the
prototype:

\begin{verbatim}
double
my_screening_function(
  double d_t9,
  double d_rho,
  double d_ye,
  int i_Z1,
  int i_A1,
  int i_Z2,
  int i_A2
  void *p_my_data
);
\end{verbatim}

The routine may have any appropriate name.  The necessary inputs are:

\begin{itemize}

\item {\bf d\_t9:} The temperature $T_9$, that is, the temperature
in $10^9$K at which to compute the screening factor.

\item {\bf d\_rho:} The density $\rho$ in g/cc at which to compute the screening
factor.

\item {\bf d\_ye:} The electron-to-baryon ratio $Y_e$ at which to compute the
screening factor.

\item {\bf i\_Z1:}  The atomic number of the first reactant.

\item {\bf i\_A1:}  The mass number of the first reactant.

\item {\bf i\_Z2:}  The atomic number of the second reactant.

\item {\bf i\_A2:}  The mass number of the second reactant.

\item {\bf p\_my\_data:}  A pointer to a user-defined data structure containing
extra data to be passed to the user's screening function.  If there are
no extra data, this should be set to NULL.

\end{itemize}

Given these inputs, the user's routine should return the factor by which
to change the rate for the reaction $(i\_Z1,i\_A1) + (i\_Z2,i\_A2)$ to
account for screening.  Thus, if the rate for the reaction is $[12]$ in the
absence of screening, the rate for the reaction with screening should be
$f \times [12]$, where $f$ is the screening factor returned by
the user's routine.

There are two ways to call the screening routine.  The first way is to
call the routine through the libnucnet API routine
Libnucnet\_\_Net\_\_computeScreeningFactorForReaction().  The prototype
for this routine is

\begin{verbatim}
double
Libnucnet__Net__computeScreeningFactorForReaction(
  Libnucnet__Net *self,
  Libnucnet__Reaction *p_reaction,
  double d_t9,
  double d_rho,
  double d_ye,
  Libnucnet__Net__screening_function my_screening_function,
  void *p_user_data
);
\end{verbatim}

The necessary inputs are:

\begin{itemize}

\item {\bf self:} A pointer to a Libnucnet\_\_Net structure, which contains
the nuclear network (nuclear + reaction) data.

\item {\bf p\_reaction:}  A pointer to the Libnucnet\_\_Reaction structure
for the reaction whose screening factor is desired.

\item {\bf d\_t9:} The temperature $T_9$, that is, the temperature
in $10^9$K at which to compute the screening factor.

\item {\bf d\_rho:} The density $\rho$ in g/cc at which to compute the screening
factor.

\item {\bf d\_ye:} The electron-to-baryon ratio $Y_e$ at which to compute the
screening factor.

\item {\bf my\_screening\_function:} The name of the user's screening function.
Typically, this needs to be cast as a Libnucnet\_\_Net\_\_screening\_function.
Note that as of version 0.13, this type was changed from
Libnucnet\_\_Net\_\_screeningFunction to be consistent with the libnucnet
standard that a lower camel case ``method'' means the input structure is
the first argument.

\item {\bf p\_my\_data:}  A pointer to a user-define data structure containing
extra data to be passed to the user's screening function.  If there are
no extra data, this should be NULL.

\end{itemize}

The output from this routine is a double containing the screening factor
for the reaction pointed to by p\_reaction, as computed by the user's routine.
To apply this screening factor, the user then multiplies the forward and
reverse rates for the reaction by the value returned by
Libnucnet\_\_Net\_\_computeScreeningFactorForReaction().

A more useful alternative for network calculations
is to set the screening function for a particular zone and
then let Libnucnet\_\_Zone\_\_computeRates() call the user's routine
directly.  To do this, the user calls the API routine
Libnucnet\_\_Zone\_\_setScreeningFunction(), which has the prototype

\begin{verbatim}
void
Libnucnet__Zone__setScreeningFunction(
  Libnucnet__Zone *self,
  Libnucnet__screening_function my_screening_function,
  void *p_user_data
);
\end{verbatim}

The inputs to this routine are
\begin{itemize}

\item {\bf self:}  A pointer to the Libnucnet\_\_Zone under consideration.

\item {\bf my\_screening\_function:}  The name of the user's screening
function.  The user should cast this as a Libnucnet\_\_screening\_function.

\item {\bf p\_my\_data:}  The pointer to the user's data structure.  If there
are no extra data to pass to the user's function, this should be NULL.
\end{itemize}

The user must set the screening function prior to calling
Libnucnet\_\_Zone\_\_computeRates(), otherwise the screening function will
not be called.  To clear the screening function, the user simply calls
the API routine Libnucnet\_\_Zone\_\_clearScreeningFunction().

Libnucnet computes the screening factor for reactions with more than two
reactants in the traditional manner of viewing the full reaction as a sequence
of intermediate reactions.  For example, for the three-body reaction
$a + b + c$, libnucnet uses the user's screening routine to compute
$F_{screen}(a,b)$, the screening factor for the reaction $a + b$.
It then uses the
user's screening routine to compute $F_{screen}(a+b,c)$,
the screening factor for
the reaction $(a + b) + c$.  The total screening factor applied to the
reaction $a + b + c$ is then $F_{screen}(a,b) \times F_{screen}(a+b,c)$.
For four reactants,
that is, the reaction $a + b + c + d$, the total screening factor would
be $F_{screen}(a,b) \times F_{screen}(a+b,c) \times F_{screen}(a+b+c,d)$.
Libnucnet loops over reactants,
so this generalizes to any number of reactants.

After the screening factor $F_{screen}$ is computed for the forward reaction,
it is applied to both the forward and reverse reaction rates; thus, both the
forward and reverse rates are multiplied by $F_{screen}$.
It is possible to retrieve the screening factor applied to a reaction
in a particular zone by calling the API routine
Libnucnet\_\_Zone\_\_getScreeningFactorForReaction().

By multiplying both forward and reverse rates for a reaction by the
computed screening factor,
the network will tend to evolve at constant temperature and density and
in the absence of weak interaction rates to the same nuclear statistical
equilibrium (NSE) that it would have
without screening applied.  This may not be consistent with the physical
nature of the screening.  In particular, the presence of the electrons
may alter the binding energy of the nuclei and hence the resulting NSE.
If the user wishes to correct the reverse
reaction rate so that the NSE is consistent with the screening due to the
electrons, he or she must apply the reverse ratio correction factor
function as described in the next section.

\section{Reverse Ratio Correction Factor}

As discussed above, libnucnet applies the result of the user's screening
function to both the forward and reverse rates for a reaction.  This means
that the network abundances will tend to evolve to the same NSE that they
would without application of the screening (albeit at a different overall
rate).  If this is not consistent with the user's screening model, he or she
must apply a correction factor to the reverse ratio.  This correction factor
is in fact derived from the
factor by which one would multiply the NSE abundance for
a species to account for the effect of the electrons (or another effect).

We begin by explaining
what is meant by a libnucnet NSE correction factor.
The condition for
NSE is
\begin{equation}
\mu_i = Z_i \mu_p + N_i \mu_n, \label{eq:nse_condition}
\end{equation}
where $\mu_i$ is the chemical potential for species $i$ and $\mu_n$, $\mu_p$
are the chemical potentials for neutrons and protons, respectively, and
$Z_i$ and $N_i$ are the atomic number and neutron number for species $i$, 
respectively.  The libnucnet default is to consider all nucleons and nuclei
as ideal, classical Maxwell-Boltzmann particles; thus,
\begin{equation}
\mu_i = m_i c^2 + kT \ln\left(\frac{Y_i}{Y_{Qi}}\right)
\equiv m_i c^2 + \mu_i',\label{eq:classical_mu}
\label{eq:mui}
\end{equation}
where the quantum abundance $Y_{Qi}$ is defined as
\begin{equation}
Y_{Qi} \equiv \frac{G_i}{\rho N_A}
\left(\frac{m_i kT}{2\pi\hbar^2}\right)^{3/2}.
\label{eq:yqi}
\end{equation}
In Eq. (\ref{eq:mui}) $Y_i$ is the abundance of $i$ per nucleon and
$m_ic^2$ is the rest mass energy of species $i$.
In Eq. (\ref{eq:yqi})
$N_A$ is Avogadro's number,
$G_i$ is $i$'s nuclear partition function, and $k$ is Boltzmann's constant.
It is useful to consider that $Y_{Qi}$ is the abundance per nucleon of species
$i$ if there were one such particle in a cube with side
one thermal de Broglie wavelength in length.
The NSE abundance of species $i$ is then
\begin{equation}
Y_{i,NSE} = Y_{Qi}\exp\left\{Z_i \frac{\mu_p'}{kT} + N_i \frac{\mu_n'}{kT}
+ \frac{B_i}{kT}\right\}, \label{eq:y_i_nse}
\end{equation}
where $B_i$ is the nuclear binding energy of species $i$:
\begin{equation}
B_i = Z_i m_pc^2 + N_i m_nc^2 - m_ic^2.\label{eq:binding}
\end{equation}
As of version 0.7 of libnucnet, the quantities $Y_{Qi}$ and $B_i$ can be
computed from the API routines
{\em Libnucnet\_\_Species\_\_computeQuantumAbundance()}
and
{\em Libnucnet\_\_Nuc\_\_computeSpeciesBindingEnergy()},
respectively.

The libnucnet NSE correction factor is a factor $f_{i,corr}$ that gets added
to the exponent in Eq. (\ref{eq:y_i_nse})
to correct for deviations away from the above treatment.  In particular,
suppose in the correct treatment,
$\mu_i' \neq kT \ln \left( Y_i / Y_{Qi} \right)$.  We may then write
\begin{equation}
Y_i = Y_{Qi} \exp\left\{\frac{\mu_i'}{kT} + \left[ \ln\left(\frac{Y_i}{Y_{Qi}}
\right)- \frac{\mu_i'}{kT}\right]\right\}.
\end{equation}
If we now apply the NSE condition in Eq. (\ref{eq:nse_condition}),
we find
\begin{equation}
Y_{i,NSE}' = Y_{Qi} \exp\left\{
Z_i \frac{\mu_p'}{kT} + N_i \frac{\mu_n'}{kT} + \frac{B_i}{kT} +
\left[ \ln\left(\frac{Y_i}{Y_{Qi}}
\right)- \frac{\mu_i'}{kT}\right]
\right\}, \label{eq:y_i_nse_f}
\end{equation}
where $Y_{i,NSE}'$ indicates the nuclear statistical equilibrium abundance
in the now correct
treatment.  We thus see that
\begin{equation}
Y_{i,NSE}' = Y_{i,NSE} \exp\left\{
\ln\left(\frac{Y_i}{Y_{Qi}}
\right)- \frac{\mu_i'}{kT}
\right\} \equiv Y_{i,NSE} \times e^{f_{i,corr}}, \label{eq:y_i_nse_f_2},
\end{equation}
where $Y_{i,NSE}$ refers to the equilibrium abundance computed from the
neutron and proton chemical potentials and the species binding energy.

An example will illustrate how this works.  Suppose each nuclear species
has a constant potential energy $U_i$, perhaps due to a uniform background
of electrons.  The abundance of the species is then given by
\begin{equation}
Y_i = Y_{Qi} \exp\left\{\frac{\mu_i'}{kT} - \frac{U_i}{kT}\right\}.
\end{equation}
From this, it is clear that
\begin{equation}
f_{i,corr} = -\frac{U_i}{kT}.
\end{equation}

To accomodate an NSE correction factor, the user first writes a
Libnucnet\_\_Species\_\_nseCorrectionFactorFunction() with any appropriate
name (in the case below my\_correction\_function)
which has the prototype

\begin{verbatim}
double
my_correction_function(
  Libnucnet__Species *self,
  double d_t9,
  double d_rho,
  double d_ye,
  void *p_my_data
);
\end{verbatim}

The inputs are
\begin{itemize}

\item {\bf self:}  A pointer to a Libnucnet\_\_Species.

\item {\bf d\_t9:}  The temperature $T_9$, that is, the temperature
in $10^9$ K at which to compute the correction factor.

\item {\bf d\_rho:}  The density $\rho$ in g/cc at which to compute
the correction factor.

\item {\bf d\_ye:}  The electron-to-baryon ratio $Y_e$ at which to compute
the correction factor.

\item {\bf p\_my\_data:}  A pointer to a user-defined data structure
containing any extra data to be passed into the user's routine.  This should
be NULL if no extra data are required.

\end{itemize}

This function has the Species namespace because it generally
requires only nuclear data for a particular species, along with the temperature,
density, and $Y_e$.  The output from this
routine is a double that contains the natural logarithm of the
factor by which one multiplies the
NSE abundance of a species computed from the proton and neutron
chemical potentials and the species binding energy
to account for the correction.  Thus, in
the case of electron screening, if the
abundance of species $i$ in NSE in the absence of the electron effects is
$Y_{i,NSE}$, the user's
Libnucnet\_\_Species\_\_nseCorrectionFactorFunction
should return
the quantity $f_{i,corr}$ such that the NSE abundance with the effect of the
electrons included is $\exp(f_{i,corr}) \times Y_{i,NSE}$.

As with the screening routine, the user may call his or her routine in two
ways.  First, the user may call the libnucnet API routine\\
Libnucnet\_\_Net\_\_computeReverseRatioCorrectionFactorForReaction()
directly.  The prototype for this reaction is:

\begin{verbatim}
double
Libnucnet__Net__computeReverseRatioCorrectionFactorForReaction(
  Libnucnet__Net *self,
  Libnucnet__Reaction *p_reaction,
  double d_t9,
  double d_rho,
  double d_ye,
  Libnucnet__Species__nseCorrectionFactorFunction
    my_correction_function,
  void *p_my_data
);
\end{verbatim}

The inputs are
\begin{itemize}

\item {\bf self:} A pointer to a Libnucnet\_\_Net structure, which contains
the nuclear network (nuclear + reaction) data.

\item {\bf p\_reaction:}  A pointer to the Libnucnet\_\_Reaction structure
for the reaction whose reverse ratio correction factor is desired.

\item {\bf d\_t9:}  The temperature $T_9$, that is, the temperature
in $10^9$ K at which to compute the correction factor.

\item {\bf d\_rho:}  The density $\rho$ in g/cc at which to compute
the correction factor.

\item {\bf d\_ye:}  The electron-to-baryon ratio $Y_e$ at which to compute
the correction factor.

\item {\bf my\_correction\_function:}  The
name of the user's correction factor function.  The user should cast this
as a Libnucnet\_\_Species\_\_nseCorrectionFactorFunction.

\item {\bf p\_my\_data:}  A pointer to a user-defined data structure
containing any extra data to be passed into the user's routine.  This should
be NULL if no extra data are required.

\end{itemize}

The routine loops over all reactants and products in the reaction and
computes their NSE correction factors.  From these, the routine then
returns the double by which the reverse reaction rate should be multiplied
to account for the electron screening effects.

The second way to allow libnucnet to call the user-supplied correction
factor routine is to set the correction factor for a given zone and
let the\\
Libnucnet\_\_Zone\_\_computeRates() routine call it.
To do this, the user calls the routine
Libnucnet\_\_Zone\_\_setNseCorrectionFactorFunction(),
which has the prototype

\begin{verbatim}
void
Libnucnet__Zone__setNseCorrectionFactorFunction(
  Libnucnet__Zone *self,
  Libnucnet__Species__nseCorrectionFactorFunction
    my_correction_function,
  void *p_my_data
);
\end{verbatim}

The inputs to this routine are
\begin{itemize}

\item {\bf self:}  A pointer to the Libnucnet\_\_Zone under consideration.

\item {\bf my\_correction\_function:}  The name of the user's correction
factor function.  The user should cast this as a
Libnucnet\_\_Species\_\_nseCorrectionFactorFunction.

\item {\bf p\_my\_data:}  The pointer to the user's data structure.  If there
are no extra data to pass to the user's function, this should be NULL.
\end{itemize}

The routine uses the user's routine to compute the reverse ratio correction
factor $f_{corr}$ and applies it to the reverse rate for each reaction. 

\section{Corrections and NSE}
In order to be clear about the procedure for applying the screening and
reverse ratio correction factors, we provide an example.  Consider the
reaction
\begin{equation}
^{12}{\rm C} + \alpha \to ^{16}{\rm O} + \gamma. \label{eq:reaction}
\end{equation}
The contribution of this reaction
to $dY({^{12}{\rm C}})/dt$, the time rate of change of the abundance
of $^{12}$C, 
is
\begin{equation}
-N_A \langle \sigma v \rangle \rho Y(^{12}{\rm C}) Y(\alpha ) + \lambda_\gamma
Y(^{16}{\rm O}),  \label{eq:diffeq}
\end{equation}
where $N_A$ is Avogadro's number, $\langle \sigma v \rangle$ is the
thermally-averaged interaction cross section for a reaction between 
a $^{12}$C nucleus and an $\alpha$ particle, $\rho$ is the mass density,
and $\lambda_\gamma$ is the rate per second for an $^{16}$O nucleus to
disintegrate back into a $^{12}$C and an $\alpha$.

If the system attains NSE, then detailed balance ensures that forward and
reverse reaction flows come into balance.  Thus, 
\begin{equation}
N_A \langle \sigma v \rangle \rho Y_{NSE}(^{12}{\rm C}) Y_{NSE}(\alpha )
= \lambda_\gamma Y_{NSE}(^{16}{\rm O}),  \label{eq:diffeqNse}
\end{equation}
This means that the ``reverse ratio'' $R$ relating the forward and reverse
reaction rates is
\begin{equation}
R = \frac{\lambda_\gamma}{N_A \langle \sigma v \rangle} = 
\frac{\rho Y_{NSE}(^{12}{\rm C}) Y_{NSE}(\alpha )}{Y_{NSE}(^{16}{\rm O})}.
\label{eq:RevRatio}
\end{equation}
In this particular case, we can apply Eq. (\ref{eq:y_i_nse}) to find
\begin{equation}
R = \rho\frac{Y_{Q,^{12}{\rm C}} Y_{Q,\alpha}}{Y_{Q,^{16}{\rm O}}}
\exp\left\{\frac{B_{^{12}{\rm C}}}{kT} + \frac{B_{\alpha}}{kT}
- \frac{B_{^{16}{\rm O}}}{kT}\right\}.
\label{eq:RevRatio2}
\end{equation} 
If we compute reverse rates from forward rates using reverse ratios computed
from detailed balance in this way, our network will tend to evolve at
constant temperature and density and
in the absence of weak interaction rates to the expected NSE.

We now compute screening.  Suppose the screening factor for the forward reaction
in Eq. (\ref{eq:reaction}) is $F_{screen}(\alpha,\gamma)$ and for the reverse
reaction is $F_{screen}(\gamma,\alpha)$.
If we apply these screening factors, we find
\begin{equation}
R' = \frac{F_{screen}(\gamma,\alpha) \lambda_\gamma}{F_{screen}(\alpha,\gamma) N_A \langle \sigma v \rangle } =
\frac{F_{screen}(\gamma,\alpha)}{F_{screen}(\alpha,\gamma)}
\times R.
\label{eq:RevRatioNew}
\end{equation}

Since libnucnet computes a reverse rate from a forward rate using detailed
balance, the effect of screening on the reverse rate is obtained from
consideration of the NSE achieved.
As mentioned above,
the user-supplied NSE correction factor function
computes the factor by which $Y_{i,NSE}$, the
NSE abundance of species $i$,
changes due to the screening electrons; thus, if the user's routine
returns the correction factor $f_{i,corr}$ for species $i$, then
\begin{equation}
Y_{i,NSE}' = e^{f_{i,corr}} \times Y_{i,NSE},
\label{eq:newNse}
\end{equation}
where the prime indicates correction for the screening electrons.  Because
of detailed balance, we thus know
\begin{equation}
R' =
\frac{\rho Y_{NSE}'(^{12}{\rm C}) Y_{NSE}'(\alpha )}{Y_{NSE}'(^{16}{\rm O})}
= e^{f_{^{12}{\rm C},corr} + f_{\alpha,corr} - f_{^{16}{\rm O},corr}} \times
\frac{\rho Y_{NSE}(^{12}{\rm C}) Y_{NSE}(\alpha )}{Y_{NSE}(^{16}{\rm O})}
\label{eq:RevRatioPrime}
\end{equation}
Comparison of Eqs. (\ref{eq:RevRatio}), (\ref{eq:RevRatioNew}),
and (\ref{eq:RevRatioPrime}) shows that
\begin{equation}
F_{screen}(\gamma,\alpha) =
e^{f_{^{12}{\rm C},corr} + f_{\alpha,corr} - f_{^{16}{\rm O},corr} } \times
F_{screen}(\alpha,\gamma).
\end{equation}
We may write
\begin{equation}
R' = F_{corr} \times R.
\end{equation}
$F_{corr}$ is the reverse ratio correction factor, the quantity returned by\\
\\
Libnucnet\_\_Net\_\_computeReverseRatioCorrectionFactorForReaction()\\
\\
For the reaction in Eq. (\ref{eq:reaction}), it is clear that
\begin{equation}
F_{corr} = e^{f_{^{12}{\rm C},corr} + f_{\alpha,corr} - f_{^{16}{\rm O},corr}}.
\end{equation}
In general,
\begin{equation}
F_{screen}({\rm reverse}) = F_{corr} \times F_{screen}({\rm forward}).
\end{equation}

Libnucnet applies the user's screening function, that is, his or her\\
\\
Libnucnet\_\_screening\_function()\\
\\
to compute the screening
factor for the forward rate and then the user's\\
\\
Libnucnet\_\_Species\_\_nseCorrectionFactorFunction()\\
\\
on the various reactants to compute the reverse ratio correction factor
and the screening factor for the
reverse reaction.  With this treatment, the system will evolve in the
absence of weak reactions and at constant temperature and
density to the NSE appropriate
to the user's NSE correction factor function [that is, once the abundances
reach NSE, they will be given by Eq. (\ref{eq:newNse})].  Note that if the
user does not supply an NSE correction factor function, the forward and
reverse ratios have the same screening factor.
This means the abundances will achieve the same NSE they would have in the
absence of screening although the time to reach that equilibrium would
typically be shorter.


\begin{thebibliography}{1}

\bibitem{1982ApJ...258..696W}
{\sc R.~K. {Wallace}, S.~E. {Woosley}, and T.~A. {Weaver}}, {\em The
  thermonuclear model for x-ray transients}, \apj, 258 (1982), pp.~696--715.

\end{thebibliography}


\end{document}
